\documentclass{article} % Or a suitable recipe-specific document class

% Packages for formatting and any extra symbols
\usepackage{graphicx} % If you want to add images
\usepackage{amsmath}  % For nicer fractions, if needed

\title{Tangy Eggplant Sandwich}
\author{Love Grover}
\date{}
\begin{document}

\maketitle

\section*{Recipe Overview}
This simple yet incredibly flavorful sandwich highlights the deliciousness of eggplant, combined with tangy, homemade sour curd, Indian spices, and the richness of ghee.  Feel free to adjust spice levels to your liking!  

Yields: 1 sandwich (plus extra sour curd).
Prep Time: 15 minutes (plus curd setting time).
Cooking Time: 10 minutes.

\section*{Ingredients}

\begin{description}
\item[For the Sour Curd] \hfill

\begin{itemize}
\item 1/2 cup warm whole milk (not boiling)
\item 1 tablespoon plain yogurt (store-bought, with live cultures)
\end{itemize}

\item[For the Sandwich]\hfill

\begin{itemize}
\item 2 thick slices of your favorite bread (sourdough and whole grain work especially well)
\item 1/4 cup sliced eggplant  (approximately 1/4-inch thick slices)
\item 1 tablespoon ghee, plus more for the bread
\item 1/4 cup prepared sour curd (see instructions)
\item 1/4 cup finely chopped onion
\item 1/4 teaspoon aamchur (dry mango powder)
\item 1/4 teaspoon ground coriander
\item Himalayan pink salt to taste
\item Freshly ground black pepper to taste
\item 1 small green chili or jalapeno pepper, finely chopped
\item 2 slices tomato
\item 1-2 cloves garlic, crushed (optional)
\end{itemize}
\end{description}

\section*{Instructions}

\begin{description}
\item[1. Prepare the Sour Curd (at least one day in advance):]\hfill
\begin{itemize}
   \item In a small bowl, whisk together the warm milk and yogurt.
   \item Cover the bowl and place it in a warm, undisturbed spot for 24 hours or more, until the curd thickens and has a tangy flavor. The longer it sits, the tangier it will become. 
\end{itemize}

\item[2. Cook the Eggplant]\hfill
\begin{itemize}
    \item Heat 1 tablespoon of ghee in a pan over medium heat. Add the eggplant slices and sprinkle them with a bit of pink salt and pepper. Cook until softened and slightly golden, flipping halfway through. Remove from the pan and set aside.
\end{itemize}


\item[3. Make the Spice Mix]\hfill
\begin{itemize}
    \item Combine the prepared sour curd, chopped onion, aamchur, ground coriander, pink salt, black pepper, and green chili. Mix well.
\end{itemize}


\item[4. Toast the Bread (with optional garlic)]\hfill
\begin{itemize}
    \item Lightly spread ghee on both sides of your bread slices. Add crushed garlic to the ghee for extra flavor, if desired. Toast the bread in a separate pan over medium heat until golden brown and crispy.
\end{itemize}


\item[5. Assemble the Sandwich]\hfill
\begin{itemize}
\item On one slice of bread, spread a generous layer of the curd mixture. Top with the cooked eggplant slices and tomato slices. Place the other slice of bread on top.
\end{itemize}
\item[6. Enjoy!]\hfill
\begin{itemize}
    \item Serve your delicious sandwich immediately.
\end{itemize}
\end{description}

\section*{Some Tips} 
\begin{description}
    \item[1. For acquired tangy taste buds]\hfill
\begin{itemize}
\item  You can add more aamchur (mango powder) directly to the sandwich filling and/or let the sour curd ferment for a longer time (2-3 days) for an extra tangy kick!
\end{itemize}
\end{description}
\end{document}
